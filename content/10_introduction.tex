\chapter{Introduction}
\label{sec:intro}

% Die Einleitung schreibt man zuletzt, wenn die Arbeit im Großen und
% Ganzen schon fertig ist. (Wenn man mit der Einleitung beginnt - ein
% häufiger Fehler - braucht man viel länger und wirft sie später doch
% wieder weg). Sie hat als wesentliche Aufgabe, den Kontext für die
% unterschiedlichen Klassen von Lesern herzustellen. Man muß hier die
% Leser für sich gewinnen. Das Problem, mit dem sich die Arbeit befaßt,
% sollte am Ende wenigsten in Grundzügen klar sein und dem Leser
% interessant erscheinen. Das Kapitel schließt mit einer Übersicht über
% den Rest der Arbeit. Meist braucht man mindestens 4 Seiten dafür, mehr
% als 10 Seiten liest keiner.


It is known that problem of scheduling parallel programs onto
multiprocessor computer in its general form is
NP-complete~\cite{Ullman1975}. There are know constrained models,
which allow to determine optimal schedule in polynomial
time~\cite{Hu1961, Coffman1972, Papadimitriou1979}.  But simplicity of
these models barely allows their utilization for scheduling of real
parallel applications.

Parallelization of an application requires extensive knowledge about
program structure. There exist a variety of methods to declare program
structure, that is convenient for the programmer, but also allows to
effectively schedule the application on multiprocessor system. Such
parallelization methods among others are fork-join pattern[links],
message-passing systems[links], map-reduce frameworks[links],
future-based parallelism[links], asynchronous lambda
approaches[links].

These methods differ in their API, in their granularity and
functionality. But all they have one common idea: order sequential
sections of the application in such temporal order that respects their
mutual dependencies. This process is called
\emph{scheduling}. Sequential sections represent parts of actual
execution trace of the program and have corresponding starting and
finish time. The same part of the application code can appear as
several sequential sections.

Dependencies between sequential sections represent transfers of data
that is calculated in precedent sections and used as an input by
subsequent ones. Relations between sequential sections can be modeled
as a graph, and thus all the dependencies go from past to future this
graph has no cycles. Thus, such model is called Directed Acyclic Graph
(DAG).

DAG is a common and well-studied way to represent the execution of
parallel application \cite{zheng20131673,
  Blumofe:1996:ADD:237502.237574}.


\section{Motivation}

\section{Document Structure}



% Referencing other chapters: \ref{sec:state} \ref{sec:design}
% \ref{sec:implementation} \ref{sec:evaluation} \ref{sec:futurework}
% \ref{sec:conclusion}

% \begin{table}[htp]
%   \centering
%   \begin{tabular}{lrr}
%     \textbf{Name} & \textbf{Y} & \textbf{Z} \\
%     \hline
%     \textit{Foo} & 20,614 & \unit[23]{\%} \\
%     \textit{Bar} & 9,914 & \unit[11]{\%} \\
%     \textit{Foo + Bar} & 30,528 & \unit[34]{\%} \\
%     \hline
%     \textit{total} & 88,215 & \unit[100]{\%} \\

%   \end{tabular}
%   \caption[Some interesting numbers]{Various very important looking numbers and sums.}
%   \label{tab:numbers}
% \end{table}

% More text referencing Table~\ref{tab:numbers}.

% \section{Another Section}

% \begin{figure}[tbp]
%   \centering
%   \includegraphics[width=0.8\textwidth]{images/squirrel}
%   \caption[Short description]{A long description of this squirrel figure.
%   Image taken from
%   \url{http://commons.wikimedia.org/wiki/File:Sciurus-vulgaris_hernandeangelis_stockholm_2008-06-04.jpg}}
%   \label{fig:squirrel}
% \end{figure}

% Citing \cite{bellard2005qfa} other documents \cite{bellard2005qfa, boileau06}
% and Figure~\ref{fig:squirrel}.

% Something with umlauts and a year/month date:
% \cite{becher04:_feurig_hacken_mit_firew}.

% And some online resources: \cite{green04}, \cite{patent:4819234}

% \section{Yet Another Section}

% \todo{add content}

% \begin{figure}[tbp]
%  \missingfigure{Come up with a mindblowing figure.}
%  \caption{A mindblowing figure}
%  \label{fig:todo}
% \end{figure}

% \section{Test commands}

% \drops \LLinux \NOVA \QEMU
% \texttt{memcpy}
% A sentence about BASIC. And a correctly formatted one about ECC\@.

\cleardoublepage

%%% Local Variables:
%%% TeX-master: "diplom"
%%% End:
